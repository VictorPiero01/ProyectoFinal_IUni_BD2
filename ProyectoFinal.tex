\documentclass[12pt,a4paper,oneside]{book} 
\usepackage[utf8]{inputenc}
\usepackage[spanish]{babel}
\usepackage{amsmath}
\usepackage{amsfonts}
\usepackage{amssymb}
\usepackage{abstract} % Allows abstract customization
\renewcommand{\abstractnamefont}{\normalfont\bfseries} % Set the "Abstract" text to bold
\renewcommand{\abstracttextfont}{\normalfont\small\itshape} % Set the abstract itself to small italic text


\usepackage{amssymb, amsmath}
\usepackage{graphicx}
\usepackage[left=2.54cm,right=2.54cm,top=2.54cm,bottom=2.54cm]{geometry}

\begin{document}
	
	\thispagestyle{empty} 
	
	\begin{center} 
		\LARGE{UNIVERSIDAD PRIVADA DE TACNA} \\[0.5cm] 
		\Large{FACULTAD DE INGENIERÍA DE SISTEMAS}\\[0.5cm] 
		\large{ ESCUELA PROFESIONAL DE INGENIERÍA SISTEMAS} 
	\end{center}
	
	\begin{figure}[htb]
		\centering \includegraphics[width=6cm, height=7cm]{img/uptlogo.jpg}
	\end{figure}
	
	\begin{center} 
			\LARGE{\bf PROYECTO TRABAJO FINAL \newline UNIDAD I }\\ \vspace{.25cm}
		
	\end{center}

	\begin{center} 
		
		\textbf {CURSO}\\ 
		\large Base de Datos II \\
		
		\textbf {DOCENTE}\\
		\large Mag. Patrick Cuadros Quiroga\\
	
		\textbf {INTEGRANTES}\\
		\large Tarqui Montalico, Risther Jaime - 2017057857 \\
		\large Limache Victorio, Victor Piero - 2017057857 \\
		\large Sanchez Rodriguez, Bayron - 2017057857 \\
		\large Liendo Velasquez, Joaquin - 2017057857 \\
		\large Callata Flores, Rafael - 2017057857 \\
		
	\end{center}

	
	
	\begin{center} 
		\Large \textsc{Tacna - Perú} \\
		\Large \textsc{2020 } 
	\end{center}

	\newpage
	
	%%INICIO abstract
	\begin{abstract}
		En el presente analisis se pretende realizar de forma concisa y clara las necesidades del cliente  en términos de software que se realizará.En esta documentación se plasmará los requerimientos que servirán de guía para desarrollar el software en sus distintas etapas, ayudándonos a validar e inspeccionar la construcción de este, aplicando la calidad de Software.Por ello se trabajara con  Sonarqube para el análisis de código estático para analizar el código y encontrar errores de código, y vulnerabilidades de seguridad.El análisis Codigo de SonarSource tiene una gran cobertura de estándares de calidad bien establecidos.\\
		\begin{center}
			
			\textbf{Abstract}
		\end{center}
		In this analysis it is intended to carry out in a concise and clear way the needs of the client regarding the software to be produced, this documentation will establish the requirements that will serve as a guide for the development of the software in its different stages, helping us to validate and inspect the construction of this, applying the quality of the Software. For this reason, we will work with Sonarqube for static code analysis to analyze the code and find code bugs and security vulnerabilities. The SonarSource Code analysis has a great coverage of well established quality standards \\
	\end{abstract}
	%%FIN abstract
	
	\newpage
	
	\begin{center} 
		\LARGE{\bf PROYECTO TRABAJO FINAL \newline UNIDAD I }\\ \vspace{.25cm}
	\end{center}

	
	\begin{enumerate}
		
		\item Antecedentes o introducción.
		
		\item Titulo.  
		
		\item Planteamiento del problema.  
				\begin{enumerate}
					\item Problema
					\item Justificación
					\item Alcance
				\end{enumerate}
		\item Referentes teóricos.  
		
		\item Objetivos.  
				\begin{enumerate}
					\item General
					\item Específicos
				\end{enumerate}
		\item Referentes teóricos.
		  
		\item Desarrollo de la propuesta. 
			\begin{enumerate}
				\item Tecnología de información 
				\item Metodología, técnicas usadas
				\item Aplicación
			\end{enumerate}
			
		  
	\end{enumerate}
	
	

\end{document}
